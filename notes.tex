\documentclass[11pt,oneside]{amsbook}

\title{Real and linear analysis}
\author{Course notes\\Based on material from\\``Measure theory'' by Terence Tao, and\\``Real analysis'' by Bruckner et al}

\usepackage[vscale=.8,vmarginratio=4:3]{geometry}
\usepackage{mathpazo,amssymb}
\usepackage{setspace}\onehalfspacing\raggedbottom
\renewcommand{\labelitemi}{$\circ$}
\renewcommand{\labelenumi}{(\alph{enumi})}
\renewcommand{\chaptername}{Part}
\renewcommand{\thechapter}{\Roman{chapter}}
\usepackage{remreset}
\makeatletter\@removefromreset{section}{chapter}\makeatother
\usepackage{etoolbox}
\makeatletter
\pretocmd{\@seccntformat}{\S}{}{}
\patchcmd{\tocsection}{#2.}{\S#2.}{}{}
\apptocmd{\tocsection}{\dotfill}{}{}
\makeatother
\usepackage[linktoc=all]{hyperref}
\usepackage{tikz}

\newcommand{\set}[1]{\left\{\,#1\,\right\}}
\newcommand{\NN}{\mathbb N}
\newcommand{\ZZ}{\mathbb Z}
\newcommand{\QQ}{\mathbb Q}
\newcommand{\RR}{\mathbb R}
\DeclareMathOperator{\len}{len}
\DeclareMathOperator{\vol}{vol}
\DeclareMathOperator{\dom}{dom}
\DeclareMathOperator{\rng}{rng}

\theoremstyle{definition}
\newtheorem{exerc}{Exercise}[section]
\swapnumbers
\theoremstyle{plain}
\newtheorem{thm}{Theorem}[section]
\newtheorem{cor}[thm]{Corollary}
\newtheorem{lem}[thm]{Lemma}
\newtheorem{prop}[thm]{Proposition}
\theoremstyle{definition}
\newtheorem{defn}[thm]{Definition}
\theoremstyle{remark}
\newtheorem{rem}[thm]{Remark}
\newtheorem{example}[thm]{Example}
\newtheorem*{notes}{Notes and further reading}
\newtheorem*{reading}{Reading}
\numberwithin{equation}{section}
\numberwithin{figure}{section}
\renewcommand{\theequation}{\arabic{section}.e\arabic{equation}}
\renewcommand{\thefigure}{\arabic{section}.f\arabic{figure}}

\begin{document}

\maketitle

\tableofcontents

%%%%%%%%%%%%%%%%%%%%%%%%%%%%%%%%%%%%%%%%%%%%%%%%%%
%%%%%%%%%%%%%%%%%%%%%%%%%%%%%%%%%%%%%%%%%%%%%%%%%%
\chapter{Measure theory}
%%%%%%%%%%%%%%%%%%%%%%%%%%%%%%%%%%%%%%%%%%%%%%%%%%
%%%%%%%%%%%%%%%%%%%%%%%%%%%%%%%%%%%%%%%%%%%%%%%%%%

%%%%%%%%%%%%%%%%%%%%%%%%%%%%%%%%%%%%%%%%%%%%%%%%%%
\section{The measure problem}
%%%%%%%%%%%%%%%%%%%%%%%%%%%%%%%%%%%%%%%%%%%%%%%%%%

\begin{reading}
  Tao, \S1.1 introduction, and \S1.2.3.
\end{reading}

``Measure'' is a number assigned to a set which represents its size. There are many senses in which we may mean size. Some of these are length, area, volume, mass, and even probability. Note that other senses of size such as cardinality, diameter, and density are not usually associated with measure.

The problem of finding a measure thus sounds geometric. But given our models of space, in which coordinate axes are indexed by infinitesimal points drawn from the real number system, the problem really turns out to be analytic. (In particular, this means lots of $\epsilon$'s will show up!)

The classical problem of finding a measure can be made into a formal mathematical question as follows: Does there exist a measure function $m$ which assigns to each subset $A\subset\RR$ a value $m(A)\in[0,\infty]$ satisfying:
\begin{enumerate}
\item (normality) $m(I)=$ the length of $I$ for every interval $I$;
\item (translation-invariance) $m(x+A)=m(A)$ for every $A$; and
\item (countable additivity) $m(\bigcup A_n)=\sum m(A_n)$ for every seqence of pairwise disjoint sets $A_n$.
\end{enumerate}

Perhaps surprisingly, no such measure function exists! The three very natural properties (a)--(c) actually turn out to be mutually inconsistent.

\begin{thm}[Vitali]
  There exists a set $A\subset\RR$ such that no measure can be assigned to $A$ consistently with (a)--(c).
\end{thm}

\begin{proof}
  We will consider just the unit interval $[0,1)$ with addition modulo $1$. If there is a measure $m$ on all subsets of $\RR$, then by conditions (b) and (c), $m$ restricts to a measure on subsets of $[0,1)$ which satisfies (b) with respect to addition modulo $1$.

  Now let $\QQ_1$ denote the rationals of $[0,1)$, that is, $\QQ_1=\QQ\cap[0,1)$, and consider the collection of additive cosets of $\QQ_1$ inside $[0,1)$. The cosets are of the form $a+\QQ_1$ where again addition is interpreted modulo $1$. We now let $A\subset[0,1)$ denote a system of coset representatives for this collection.

  Now every number in $[0,1)$ can be written uniquely as $a+q$ for $a\in A$ and $q\in\QQ_1$. This means that the collection of translates of $A$ by elements $q\in\QQ_1$ covers all of $[0,1)$. In particular, by (a) the measure of $\bigcup_{q\in\QQ_1}(A+q)$ is exactly $1$.

  On the other hand, by (b) and (c) we have that
  \[m\left(\bigcup\nolimits_{q\in\QQ_1}(A+q)\right)
  =\sum\nolimits_{q\in\QQ_1}m(A+q)=\sum\nolimits_{q\in\QQ_1}m(A)
  \]
  By the previous paragraph, the left-hand side of the above equation is $1$. On the other hand the right-hand side is an infinite sum of some nonnegative constant, and hence must be either $0$ or $\infty$. This is a contradiction!
\end{proof}

We remark that it is possible to modify the argument to apply directly to a measure on $\RR$ rather than going via the unit interval with addition modulo $1$. See Tao for this version.

The lesson is that we must weaken our demands on a measure $m$. Dropping condition (a) can lead to trivial measures. Dropping conditon (b) leads to several interesting problems in set theory. Weakening condition (c) to finite additivity leads to interesting solutions, but only in dimensions $\leq2$. (In dimensions $\geq3$ the Banach--Tarski paradox again gives a contradiction.)

Yet the simplest path forward (and the one that we take) is to drop the condition that \emph{every} set be measurable. The set $A$ constructed in Vitali's proof is very artificial and isn't likely to occur in any of the most commin analytical applications (see the notes below). We will simply drop the requirement that $A$ and other sets like it be in the domain of $m$. In the end, our measure function have a domain which is a proper subset of $\mathcal P(\RR)$ but still contains a rich class of sets. And it will satisy properties (a)--(c) for all the sets in its domain.

Of course we are also interested in the measure problem for subsets $\RR^n$. It can be formulated in just the same way, with condition (a) replaced by the condition that the measure of a box is equal to its volume. And a Vitali-type result can also easily be established for this version of the measure problem.

In the next section, we will begin this process by taking a step backwards and build measures with much smaller domains, and satisfying just fragments of (a)--(c).

\begin{notes}
  The proof of Vitali's theorem requires the Axiom of Choice. Specifically, it is needed to find a system of coset representatives for an uncountable collection. Solovay showed that the use of AC is essential, and that it is consistent with $\neg$AC that every set is measurable.
\end{notes}

%%%%%%%%%%%%%%%%%%%%%%%%%%%%%%%%%%%%%%%%%%%%%%%%%%
\section{Elementary measure}
%%%%%%%%%%%%%%%%%%%%%%%%%%%%%%%%%%%%%%%%%%%%%%%%%%

\begin{reading}
  Tao, \S1.1.1
\end{reading}

In this section we define a very simple measure on $\RR^n$ which is capable just of measuring a very simple type of set. The development serves two purposes: first it reveals some of the techniques which we will use later, and second it will be used explicitly in the construction of more powerful measures.

Recall that an \emph{interval} is any subset of $\RR$ of the form $(a,b)$, $[a,b)$, $(a,b]$, or $[a,b]$. We shall use the term \emph{box} for any subset of $\RR^n$ which is a Cartesian product of intervals.

\begin{defn}
  A subset $E\subset\RR^n$ is \emph{elementary} if it can be expressed as a union of finitely many boxes.
\end{defn}

In defining elementary measure, we of course assign the measure of each interval $I=(a,b)$ or $[a,b)$ or $(a,b]$ or $[a,b]$ to simply be its length; thus in all four cases $\len(I)=b-a$. Next we define the volume of a box $B=\prod I_n$ to be the product of its side lengths; thus $\vol(B)=\prod\len(I_n)$. Great.

We now wish to define the measure of an elementary set to be the sum of the finitely many boxes it is composed of. However there are two issues with this: first the constituent boxes need not be disjoint, and second there is in general more than one way to express an elementary set as a union of boxes. The following two lemmas address these two issues.

\begin{lem}
  Any elementary set $E$ can be expressed as a finite union of disjoint boxes.
\end{lem}

\begin{proof}
  First assume that $E\subset\RR^1$ and that $E=\bigcup I_i$. Then by considering all endpoints of the $I_i$ in increasing order $a_1,\ldots,a_m$ it is easy to write $E$ as the union of sets of the form $(a_i,a_{i+1})$ together with sets of the form $[a_i,a_i]$ (single points). Such a union is clearly disjoint.

  In general if $E\subset\RR^n$ and $E=\bigcup B_i$ then for each dimension $d\leq n$ consider in turn the $d$th sides of the boxes $I_i^d$. Again consider the endpoints of these intervals in increasing order $a_i^d,\ldots,a_{m_d}^d$. Then we can write $E$ as a union of small boxes which are products of sets of the form $(a_i^d,a_{i+1}^d)$ or of the form $[a_i^d,a_{i+1}^d]$. Such boxes are again disjoint.
\end{proof}

\begin{lem}
  Suppose the elementary set $E$ can be expressed in two ways a a finite union of disjoint boxes: $E=\bigsqcup B_i=\bigsqcup C_j$. Then $\sum\vol(B_i)=\sum\vol(C_j)$.
\end{lem}

\begin{proof}
  We first note that $I$ is an interval with endpoints $a,b$, and if $a=a_1,a_2,\ldots,a_m=b$ is an increasing sequence then $\len(I)=\sum\len(a_i,a_{i+1})$. This is simply because the latter summation telescopes.
  
  Next if $B$ is a box whose $d$th side has endpoints $a^d,b^d$, and if $a^d=a_1^d,a_2^d,\ldots,a_{m_d}^d=b^d$ then $\vol(B)=$ the sum of all small boxes of the form $\prod(a_{i_d}^d,a_{i_d+1}^d)$. We will call the set of such small boxes a perfect grid. Intuitively, if you break a box into a perfect grid of sub-boxes, then the volume of the box is the sum of the volumes of the sub-boxes.

  Now if $B$ is a box and one expresses it as a disjoint union of sub-boxes $B=\bigsqcup B_i$, then $\vol(B)=\sum\vol(B_i)$. This is because it is possible to find a refinement of the disjoint union $B=\bigcup D_i$ where $\{D_i\}$ is a perfect grid as in the previoius paragraph, and each $B_i$ is the union of a perfect grid of sets all of which are in the collection $\{D_i\}$. Then one can simply apply the argument of the previous paragraph to $B$ and to each $B_i$.

  Finally given $E$, $B_i$, and $C_j$ as in the problem statement, one can find a third expression $E=\bigsqcup E_k$ where $\{E_k\}$ is a \emph{refinement} of both $\{B_i\}$ and of $\{C_j\}$. That is, each $B_i$ and each $C_j$ is a disjoint union of elements of $\{E_k\}$. It follows from the previous paragraph that $\sum\vol(B_i)=\sum\vol(E_k)$ and analogously that $\sum\vol(C_j)=\sum\vol(E_k)$. This completes the proof.
\end{proof}

The two lemmas together imply that it is well-defined to define the elementary measure function on elementary sets by writing $m(\bigsqcup B_i)=\sum m(B_i)$.

\begin{prop}
  The elementary measure function $m$ satisfies
  \begin{enumerate}
  \item (normality) $m(B)=\vol(B)$ for $B$ a box;
  \item (translation-invariance) $m(x+E)=m(E)$ for $E$ elementary; and
  \item (finite additivity) $m(E\cup F)=m(E)+m(F)$ for disjoint elementary $E,F$.
  \end{enumerate}
\end{prop}

Normality is clear from the defenition of $m$. The translation-invarinace is easy because it is true of length and volume, and moreover is preserved even when we take disjoint unions. The finite additivity property is again clear from the definition of $m$. We remark that $m$ satisfies countable additivity as well (restricted to elementary sets of course), but that is much more difficult and will be addressed later on.

The above three core properties imply further useful properties as well.

\begin{prop}
  The elementary measure function $m$ satisfies
  \begin{itemize}
  \item (monotonicity) $m(E)\leq m(F)$ for elementary sets $E\subset F$; and
  \item (finite subadditivity) $m(E\cup F)\leq m(E)+m(F)$ for elementary $E,F$.
  \end{itemize}
\end{prop}

These results give an essentially complete solution to the measure problem for elementary sets. It wasn't too difficult to achieve, but perhaps not as easy as one would have thought! Even so, what about measuring other simple sets such as circles, triangles, blobs, Cantor sets, and so on? In the next section we will continue on the road to doing this.


%%%%%%%%%%%%%%%%%%%%%%%%%%%%%%%%%%%%%%%%%%%%%%%%%%
\section{Jordan measure}
%%%%%%%%%%%%%%%%%%%%%%%%%%%%%%%%%%%%%%%%%%%%%%%%%%

\begin{reading}
  Tao, \S1.1.2.
\end{reading}

In the previous section we showed that the intuitive definition of area is sensible for elementary sets, but then remarked that simple shapes like polygons and circles are not elementary. It is easy to imagine extending the elementary measure to triangles by cutting, and to polygons by applying rotations, stretches, and skews. However no such operation (called \emph{affine} transformations) can help measure a circle.

Instead we will measure the circle the way it has always been done, by using \emph{approximation}. It is not hard to visualize a circle being approximated by elementary sets, using smaller and smaller boxes near the boundary. The approximation technique will help us measure most traditional geometric figures, and even many blobby thingies.

\begin{defn}
  Let $E$ be a bounded subset of $\RR^n$. First define the lower and upper Jordan forms:
  \begin{align*}
    \underline m_j&=\inf\set{m(A):A\subset E,\, A\text{ elementary}}\\
    \overline m_j&=\inf\set{m(B):E\subset B,\, B\text{ elementary}}
  \end{align*}
  Then if $\underline m_j(E)=\overline m_j(E)$ we say that $E$ is \emph{Jordan measurable}, call the common quantity the \emph{Jordan measure} of $E$, and denote it by $m(E)$.
\end{defn}

It is immediate from the definition Jordan measure extends elementary measure in the sense that they agree on the elementary sets. This means we are justified in using ``$m$'' both for the elementary and Jordan measure. Moreover, we will show that the Jordan measure inherits many of the properties of the elementary measure: normality, translation-invariance, finite additivity, monotonicity, and finite subadditivity.

The normality and translation-invariance properties hold simply because they hold for elementary measure. The finite properties will take a little more work. For instance, in order to even state the finite additivity property, we first need to establish Boolean closure: the union of measurable sets is measurable.

Before we begin these results, it will be useful to establish the following characterization of Jordan measurability. As we will be working with approximations, the following results also illustrate our first use of $\epsilon$-style analytical arguments.

\begin{lem}
  \label{lem:jordan-equiv}
  The set $E$ is Jordan measurable if and only if either of the following holds:
  \begin{itemize}
  \item For all $\epsilon>0$ there are elementary sets $A,B$ such that $A\subset E\subset B$ such that $m(B\smallsetminus A)<\epsilon$.
    % Note: for homework students should show the difference of elementary sets is elementary
  \item For all $\epsilon>0$ there is an elementary set $A$ such that $m(E\triangle A)=0$.
  \end{itemize}
\end{lem}

\begin{proof}
  We establish only the equivalence of Jordan measurability with the first item. To begin, assume that $E$ is Jordan measurable and let $\epsilon>0$ be given. By the $\underline m_j$ definition of Jordan measure, we can find an elementary set $A\subset E$ such that $m(E)-m(A)<\epsilon/2$. By the $\overline m_j$ definition of jordan measure we can find an elementary set $B$ such that $E\subset B$ and $m(B)-m(E)<\epsilon/2$. It follows that 
\[m(B\smallsetminus A)=m(B)-m(A)=(m(B)-m(E))-(m(E)-m(A))<\epsilon
\]
as desired.

  For the converse, assume that the first item is true, and let $\epsilon>0$ be arbitrary. Then we can find elementary sets $A,B$ such that $A\subset E\subset B$ and $m(B)-m(A)<\epsilon$. From the definitions of lower and upper Jordan measure, we have that $m(A)\leq\underline m_j(E)\leq\overline m_j(E)\leq m(B)$. It follows that $\overline m_j(E)-\underline m_j(E)<\epsilon$. Since $\epsilon$ was arbitrary, we may conclude that $\underline m_j(E)=\overline m_j(E)$ and therefore that $E$ is Jordan measurable.
\end{proof}

Note that in the proof, one has to be careful when making a claim such as $m(B\smallsetminus A)=m(B)-m(A)$. It is true in these cases because: the elementary sets are closed under set differences, and so all three sets are elementary, together with the finite additivity property for elementary measure.

\begin{prop}
  If $E,F$ are Jordan measurable, then so are $E\cup F$, $E\cap F$, and $E\setminus F$.
\end{prop}

\begin{proof}
  We prove only the case of $E\cup F$. Suppose that $E,F$ are Jordan measurable. By the previous lemma, we can find elementary sets $A,B,A',B'$ such that $A\subset E\subset B$, and $A'\subset F\subset B'$, and $m(B\smallsetminus A),m(B'\smallsetminus A')<\epsilon/2$. Then we have $A\cup A'\subset E\cup F\subset F\cup F'$ and using some algebra together with the finite subadditivity of elementary measure, $m(B\cup B'\smallsetminus(A\cup A'))\leq m(B\smallsetminus A)+m(B'\smallsetminus A')<\epsilon$. Again by the previous lemma, this shows that $E\cup F$ is Jordan measurable.
\end{proof}

We are now ready to establish the remaining stated properties of Jordan measure. The following result states finite additivity, and the first paragraph of its proof gives finite subadditivity. The monotonicity property follows immediately from finite additivity.

\begin{thm}
  The Jordan measure satisfies finite additivity, that is, if $E,F$ are Jordan measurable and disjoint, then $m(E\cup F)=m(E)+m(F)$.
\end{thm}

\begin{proof}
  We first show subbaditivity, that is, that $m(E\cup F)\leq m(E)+m(F)$. Let $\epsilon>0$ be given. Using the fact that $m=\overline m_j$ we can find elementary sets $B,B'$ such that $E\subset B$, $F\subset B'$, $m(B)-m(E)<\epsilon/2$, and $m(B')-m(F')<\epsilon/2$. Using the monotonicity and subadditivity properties of the elementary measure, together with the definition of Jordan measurability, we now have:
  \begin{align*}
    m(E\cup F)&=\overline m_j(E\cup F)\\
              &\leq m(B\cup B')\\
              &\leq m(B)+m(B')\\
              &< m(E)+m(F)+\epsilon
  \end{align*}
  Since $\epsilon$ was arbitrary, we achieve the desired inequality.
% question: do we need to use the outer version here?
  
  Now additionally assume that $E,F$ are disjoint, and again let $\epsilon>0$. This time using $m=\underline m_j$, we can find elementary sets $A,A'$ such that $A\subset E$, $A'\subset E'$, $m(E)-m(A)<\epsilon/2$, and $m(F)-m(A')<\epsilon/2$. Using the fact that $A,A'$ are disjoint, the finite additivity of elementary measure, and the definition of Jordan measurability, we now have:
  \begin{align*}
    m(E\cup F)&=\underline m_j(E\cap F)\\
              &\geq m(A\cup A')\\
              &=m(A)+m(A')\\
              &>m(E)+m(F)-\epsilon
  \end{align*}
  Again letting $\epsilon$ tend to $0$, we achieve that $m(E\cup F)\geq m(E)+m(F)$.
\end{proof}

While you probably have a clear idea of what the elementary sets look like, it is now time to give some examples and non-examples of Jordan measurable sets. Some simple but useful new examples are the axis-parallel triangles. Suppose $T$ is an axis-parallel triange with leg lenghs $a$ and $b$. To prove that $T$ is Jordan measurable, note that two copies of $T$ essentially make up a box with area $ab$. Using the finite additivity, this implies that the measure of $T$ is the expected $ab/2$.

To make this argument we need to know that Jordan measure is invariant under $180^\circ$ rotation, which is clear because it is true for boxes. But also since the two copies of $T$ overlap in a line segment, we also need to know that the Jordan measure of a line segment is $0$. This fact follows from the more general result below.

\begin{lem}
  \label{lem:jordan-graph}
  Let $f$ be a continuous function defined on a closed, bounded interval. Then the graph of $f$, considered as a subset of $\RR^2$, has Jordan measure $0$.
\end{lem}

\begin{proof}
  Let $I$ denote the domain of $f$. Recall that since $I$ is closed and bounded, it is \emph{compact}. Recall also that a continuous function with a compact domain is \emph{uniformly continuous}: for any $\epsilon>0$ there exists a $\delta>0$ such that for any interval $J$, $\len(J)<\delta$ implies $\len(f(J))<\epsilon$.

  So let $\epsilon>0$ be given, and choose $\delta>0$ as above. Shrinking $\delta$ if necessary, we can suppose that $\len(I)/\delta$ is an integer $k$. Partitioning $I$ into intervals $J_1,\ldots,J_k$ each of lengh $\delta$, we have that the graph of $f$ is contained in the set
  \[E=\bigcup_{i\leq k} J_i\times[\min f(J_i),\max f(J_i)]
  \]
  Note that the min and max values in the definition of $E$ exist by the extreme value theorem. Now $E$ is a union of $k$ many rectangles each of size at most $\delta\epsilon$. Thus $E$ is elementary and its measure is at most $k\delta\epsilon$. This latter value is $\len(I)\epsilon$, so the upper measure $\overline m_j$ of the graph of $f$ is at most $\len(I)\epsilon$. Taking $\epsilon\to0$, we conclude that $f$ is Jordan measurable with measure $0$.
\end{proof}

It is now not difficult to conclude that all polygons are Jordan measurable and have the expeted measure. This is because all polygons can be decomposed into a union of axis parallel triangles (possibly overlapping on their measure zero edges).

A simple example of a set which is not Jordan measurable is the set $\QQ_1=\QQ\cap[0,1]$ of rational numbers in the unit interval. Indeed the only elementary sets $A\subset\QQ_1$ are the finite sets, and so $\underline m_j(\QQ_1)=0$. And the only elementary sets $B$ such that $\QQ_1\subset B$ are of the form $[0,1]\smallsetminus F$ where $F$ is finite, and so $\overline m_j(\QQ_1)=1$.

We can conclude from these examples that Jordan measure works very well for classical geometric objects, but not very well even for simple analytic objects such as countable dense sets, the Cantor set, and so forth. To handle such sets, we will soon work to describe the Lebesgue measure, which  satisfies \emph{countable} additivity. Before going to such generality, however, we explore the connection between Jordan measure and the Riemann integral.


%%%%%%%%%%%%%%%%%%%%%%%%%%%%%%%%%%%%%%%%%%%%%%%%%%
\section{Riemann integration}
%%%%%%%%%%%%%%%%%%%%%%%%%%%%%%%%%%%%%%%%%%%%%%%%%%

\begin{reading}
  Tao, \S1.1.3.
\end{reading}

If the picture of Lemma~\ref{lem:jordan-graph} reminded you of Riemann sums, it should. Measure theory is closely connected to integration theory, as both are concerned with calculating areas of some regions. Moreover the Jordan measure corresponds neatly with the Riemann integral.

In fact we do Darboux.

\ldots

Depending on when you last studied Riemann integration, you may be more likely to recall Riemann's classical approach. This version involves a quite expansive notation:

\begin{itemize}
\item $f$ denotes a real-valued, bounded function defined on the interval $[a,b]$.
\item $x_0,x_1,\ldots,x_k$ denotes an increasing sequence of points in $[a,b]$ (they will be rectangle endpoints), where $x_0=a$ and $x_k=b$.
\item $\mathcal P$ denotes the partition of $[a,b]$ into subintervals defined by the $x_i$, that is, into subintervals $[x_{i-1},x_i]$.
\item $\delta x_i$ denotes the length of the $i$th interval, $x_i-x_{i-1}$.
\item $\|\mathcal P\|$ denotes the norm of the partition, $\max\delta x_i$.
\item $x_1^*,\ldots,x_k^*$ denotes any selection of points such that $x_i^*\in[x_i,x_{i+1}]$.
\end{itemize}

With these pieces in hand, we can define

\begin{defn}
  With $f$, $\mathcal P$, $\delta x_i$, $x_i^*$ as above, the corresponding \emph{Riemann sum} is:
  \[\mathcal R(f,\mathcal  P,x_i^*)=\sum f(x_i^*)\delta x_i
  \]
  The \emph{Riemann integral} of $f$ on $[a,b]$ is then defined by
  \[\int_a^b f=\lim_{\|\mathcal P\|\to0}\mathcal R(f,\mathcal P,x_i^*)
  \]
  provided this limit exists. Here the limit ``exists'' and equals $L$ if for all $\epsilon$ there exists $\delta$ such that for all $\mathcal P$ and $x_i^*$ we have $\|\mathcal P\|<\delta$ implies $|R(f,\mathcal P,x_i^*)-L|<\epsilon$.
\end{defn}

It is an exercise in notation to check that $f$ is Riemann integrable if and only if $f$ is Darboux integrable, and that the two integrals have the same value.


%%%%%%%%%%%%%%%%%%%%%%%%%%%%%%%%%%%%%%%%%%%%%%%%%%
\section{Lebesge measure}
%%%%%%%%%%%%%%%%%%%%%%%%%%%%%%%%%%%%%%%%%%%%%%%%%%

\begin{reading}
  Tao, \S1.2, first few pages
\end{reading}


%%%%%%%%%%%%%%%%%%%%%%%%%%%%%%%%%%%%%%%%%%%%%%%%%%
%%%%%%%%%%%%%%%%%%%%%%%%%%%%%%%%%%%%%%%%%%%%%%%%%%
\chapter{Functional analysis}
%%%%%%%%%%%%%%%%%%%%%%%%%%%%%%%%%%%%%%%%%%%%%%%%%%
%%%%%%%%%%%%%%%%%%%%%%%%%%%%%%%%%%%%%%%%%%%%%%%%%%


%%%%%%%%%%%%%%%%%%%%%%%%%%%%%%%%%%%%%%%%%%%%%%%%%%
\section{Banach space}
%%%%%%%%%%%%%%%%%%%%%%%%%%%%%%%%%%%%%%%%%%%%%%%%%%


\end{document}


